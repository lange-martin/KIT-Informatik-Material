\documentclass[11pt]{scrartcl}
\usepackage{amsmath}
\usepackage{amssymb}
\usepackage[ngerman]{babel}
\usepackage{ulem}
\usepackage{listings}
\usepackage{bm}
\usepackage{rotating}
\usepackage{array}
\usepackage{graphicx}
\usepackage{makecell}
\usepackage{xltabular}
\title{Formale Systeme 2: Theorie}
\date{\vspace{-5ex}}
\DeclareMathAlphabet{\mathpzc}{OT1}{pzc}{m}{it}

\begin{document}
\maketitle

\section{Social Choice}

\begin{xltabular}{\linewidth}{p{60mm}p{80mm}}
    \textbf{Wahlverfahren} &
    \textbf{Eigenschaften} \\ \hline
    
    Borda Count &
    Positional Scoring Rule with $m-1$ to $0$ \\ \hline

    Condorcet &
    Winner only exists sometimes \\ \hline

    Plurality Rule &
    Ballot only includes one candidate
    
    Positional Scoring Rule with $1, 0, 0, \dots$ \\ \hline

    Plurality with Run-Off &
    No-Show Paradox (violates monotonicity) \\ \hline

    Positional Scoring Rule &
    Violates Condorcet principle \\ \hline

    Copeland Rule &
    Satisfies Condorcet principle
    
    Tournament Solution \\ \hline

    Tournament Solutions &
    Majority Graph \\ \hline

    Kemeny Rule &
    Satisfies Condorcet principle
    
    Based on weighted majority graph \\ \hline

    Voting Tree (Cup Rule) &
    Satisifies Condorcet principle

    Most such rules violate neutrality \\ \hline

    Single Transferable Vote (STV) &
    No-Show Paradox (violates monotonicity) \\ \hline

    Approval Voting (AV) &
    Ballots cannot be modelled as linear orders over the set of alternatives \\ \hline

    Median Voter Rule &
    Different ballot domain: predetermined left-to-right ordering, single-peaked preferences
    
    Satisfies Condorcet principle
    
    Strategy-proof 
    
    Weakly Pareto
    
    Independence of Irrelevant Alternatives\\ \hline

    Banach-Knaster Last-Diminisher Protocol &

    Each agent is guaranteed a proportional piece \\ \hline

    Gale-Shapley Algorithm &

    Stable matching for "marriage problem" \\
\end{xltabular}

\newpage


\begin{xltabular}{\linewidth}{p{40mm}p{100mm}}
    \textbf{Theorem} &
    \textbf{Eigenschaften} \\ \hline

    May's Theorem &
    Two alternatives

    Anonymity (order of voters irrelevant)

    Neutrality (order of candidates irrelevant)

    Postive Responsiveness (winner becomes unique if ranking increases)

    $\Leftrightarrow$ Plurality Rule \\ \hline

    Young's Theorem &
    Anonymity

    Neutrality

    Reinforcement (common winner of groups is total winner)

    Continuity (repeat voters until their winner wins in total)
    
    $\Leftrightarrow$ Positional Scoring Rule \\ \hline

    Arrow's Theorem &

    Three or more alternatives

    Weakly Pareto ($b(x \succ y) = \mathcal{N} \Rightarrow y \notin F(b)$) 
    
    Independence of Irrelevant Alternatives 
    
    $\Leftrightarrow$ Dictatorship\\ \hline

    Gibbard-Satterthwaite Theorem &
    Resolute voting procedure (exactly one winner)

    Three or more alternatives

    Surjective (any candidate can win)

    Strategy-proof (result never improves for ballot with false preference)

    $\Rightarrow$ Dictatorship \\ \hline

    Black's Median Voter Theorem &
    Odd number of voters

    Single-peaked ballots

    $\Rightarrow \exists$Condorcet winner and it is elected by median voter rule \\ \hline

\end{xltabular}

\end{document}